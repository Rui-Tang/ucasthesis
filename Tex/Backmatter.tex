%---------------------------------------------------------------------------%
%->> Backmatter
%---------------------------------------------------------------------------%
\chapter{作者简历及攻读学位期间发表的学术论文与研究成果}

\section*{作者简历}

汤瑞,男,中国科学院大学2018级硕士研究生,就读于中国科学院计算技术研究所高性能中心。

\section*{申请或已获得的专利:}

\begin{enumerate}
    \item 谭光明,汤瑞,邵恩. ⼀种⼯作流作业优先级计算以及节点调度控制⽅法,中国,202010063983.8.
    \item 谭光明,汤瑞,邵恩. ⼀种具有业务快速恢复功能的容器组更新系统
\end{enumerate}

\chapter[致谢]{致\quad 谢}\chaptermark{致\quad 谢}% syntax: \chapter[目录]{标题}\chaptermark{页眉}
\thispagestyle{noheaderstyle}% 如果需要移除当前页的页眉
%\pagestyle{noheaderstyle}% 如果需要移除整章的页眉

时光荏苒,三年光阴如梭,在本文完成之际,三年研究生生涯也即将结束。在这即将毕业的时刻,我心中有着对未来工作的希冀,在这三年的时光中,在学习,工作,科研和生活方面,我都收获了巨大的成长。我心中深知我的成长不仅仅有我自己的努力,更多需要感谢导师的细心指点,母校悉心教诲,还有家人同学朋友共同地支持和帮助。所以我在此表达衷心地感谢。

首先,我要感谢我的导师—谭光明教授。谭老师学识渊博,严谨认真。在学术研究方面和工作生活方面谭老师都给予了我巨大的支持和帮助。从选择研究方向,到毕业设计开题、中期答辩。每次我遇到问题,谭老师都会在百忙之际抽出时间对我悉心教导,每次我都受益匪浅。在此对谭老师表达深深的感谢。

感谢邵恩老师,邵恩老师尽心尽力,认真辅导课题组每个学生的学习和研究。在实验室的时间邵老师总是认真负责地指点研究方向,讨论研究思路,在专利、论文、毕设开题、中期的准备过程中,邵老师总是认真地帮我修改内容。邵老师认真负责的工作态度深深地鼓舞了我。同时在生活方面邵老师也给与了很多帮助,在这里对邵老师表示衷心地感谢。

感谢实验室以及计算所所有老师,在计算所的三年时间里各位老师都对我提供了或多或少的帮助。感谢实验室所有师兄师姐师弟师妹,在这三年中我们一起努力,相互帮助,都收获了很大的成长。

感谢家人与朋友对我的关系和支持,每当我心情低落沉闷时,都是你们在身边鼓励我支持我。

最后,再次感谢各位老师和同学,感谢百忙之中参加答辩和论文评阅的老师!

\cleardoublepage[plain]% 让文档总是结束于偶数页,可根据需要设定页眉页脚样式,如 [noheaderstyle]
%---------------------------------------------------------------------------%
