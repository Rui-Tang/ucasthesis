\chapter{分布式大数据内存计算系统Spark系统实现机制概述}\label{chap:basic}
\section{Spark系统实现概述概述}
Spark是一种基于内存的分布式计算框架。Spark框架分为四层:用户层、作业管理执行层、资源任务调度层、物理执行层。在用户层,用户需要使用框架api编写应用代码,准备输入数据,配置应用参数。用户将这些信息提交给作业管理执行层后,框架会将用户应用转化为具体的作业执行计划,分为逻辑处理计划(数据单元和数据处理依赖关系)和物理执行计划(具体执行计划和阶段)。生成了具体的执行计划之后,框架的资源任务调度层会向集群资源管理器申请资源,资源申请成功后,框架把具体任务和资源绑定,并发送作业执行计划给物理执行层。物理执行层收到作业执行计划之后,会根据作业的配置信息下载应用代码,运行任务并将执行结果反馈给框架。下面会对Spark框架的四层进行进一步的介绍。

\begin{enumerate}
    \item 用户层:用户层提供了丰富的应用接口。包括面向结构化数据的查询接口Spark SQL,面向图计算的GraphX、面向机器学习的Spark MLlib,面向实时流计算的Spark Stream。用户使用提供的接口可以实现丰富的应用程序。
    \item 作业管理执行层:框架收到用户提交的应用程序后,就将其解析成Spark可以识别的逻辑执行计划。比如对于Spark SQL作业,会通过SQL语言的解释器将SQL语句解释为Spark可识别的作业执行计划,一般是以DAG(Directed Acycilc Graph)有向无环图结构组成的作业执行计划。Spark框架会遍历DAG,如果遇到需要Shuffle的边就会对当前边进行切割。遍历完毕之后整个DAG被切割成多个子图,每个子图都会更具拓扑顺序从前到后逐渐调度执行。
    \item 资源任务调度层:Spark框架一般是主从(Master-Slave)架构。主节点负责接受用户的应用,解析作业,管理执行作业,并在出错场景下做容错管理。从节点主要负责接受主节点的任务执行计划,具体执行作业并将结果汇总给主节点。
    \item 物理执行层:物理执行层会执行真正的任务。物理执行层下载或者接受得到用户代码,申请内存开始执行,将执行结果存入特定文件或者通过网络发送给下游节点。任务执行过程中需要内存,输出临时数据也需要在内存中缓存。执行结束之后框架会将执行结果发送给框架。
\end{enumerate}

\section{Spark系统整体架构}
Spark框架系统架构采用了Master-Worker结构。Master节点负责管调度理应用,Worker负责具体执行任务。Master节点会管理监控所有的Worker节点,并且调度执行应用作业,将任务调度给Worker节点,监控收集Worker节点上任务的执行状况。Worker节点会定时和Master节点进行心跳交互。收到Master节点发送的任务后后在本地执行任务,并监控任务状态,将任务执行状况上报给Master节点。具体有一下几个概念需要解释。

Spark Application,Spark应用指的是一个可运行的Spark程序。程序中包含main函数,一般流程是从数据源读取输入数据,进行一系列的处理,计算的到结果后保存到磁盘中。应用程序包含了一些配置参数,例如需要使用的CPU个数,Executor内存大小等。用户可以直接使用Spark提供的接口实现程序,也可以使用Spark SQL编写程序,Spark SQL框架会将SQL查询语言转化成一个Spark应用。

Spark Driver,Spark驱动程序是指运行Spark应用main函数的进程,它会创建Spark Context。一般是指Master节点运行的Spark应用进程,Driver进程独立于Master进程。Driver进程会通过DAGScheduler、TaskScheduler、SchedulerBackEnd等模块调度调度应用中的具体任务。

Executor,Spark执行器是Spark集群的一个资源单位。Spark Driver会向Yarn、K8s等资源调度器申请资源,资源分配成功之后会启动一个Executor进程。之后Spark Driver通过DAGScheduler和TaskScheduler调度执行作业,并通过SchedulerBackend将具体任务发送给Executor。Executor收到任务请求后会在本地JVM虚拟机创建进程执行任务。

Task,Spark计算任务。Driver程序启动后会将Spark应用切分成多个计算任务,然后调度给Executor执行。Task是最小的计算任务,Executor会给每个Task分配一个JVM进程,执行具体的计算任务,如Map算子、Filter算子、Reduce算子等。Executor一般会占用多个CPU,Task一般使用一个CPU,所以Executor中可以执行多个Task,所有Task共享Executor的内存。

这几个概念中Driver需要进一步解释。Driver包含SparkContext对象,SparkContext会将应用程序转化成一个DAG图,之后遍历DAG图,如果遇到一个action操作,也就是需要具体计算结果的操作,SparkContext就会给DAGScheduler提交一个Job。DAGScheduler收到Job请求之后会从后向前遍历作业,如果遇到边是窄依赖,就会将当前节点加入当前Stage,如果遇到的边是宽依赖,也就是需要Shuffle的边,DAGScheduler就会创建一个新的Stage,并将该节点加入新的Stage之中。遍历结束之后会得到许多个Stage,DAGScheduler会将Stage打包成TaskSet后发送给TaskScheduler执行。TaskScheduler后根据调度策略选择合适的Executor调度执行计算任务。执行结束之后Executor会讲结果汇总发送给Driver。


\section{Spark内存管理模块分析}

Spark是一种基于内存的分布式计算框架,在计算的过程中通过将数据缓存在内存中加速计算过程。所以内存管理模块是Spark系统的核心模块。Spark内存管理机制在JVM虚拟内存管理的基础上对Executor模块拥有的内存进行了进一步的分配管理。也支持在Executor外的系统内存中直接开辟内存空间用于存储计算过程中产生的大量数据。通过使用对外内存,可以减少内存空间不足导致的垃圾回收以及内存不足错误。

Spark框架执行过程中内存管理模块会申请JVM内存,申请的大小由配置参数决定。内存管理模块会将内存分为四个部分。Storage Memory,Execution Memory,User Memory,Reserved Memory。四部分内存有不同的用途。

\begin{enumerate}
    \item Storage Memory 用于缓存中间数据。比如对于一份数据data1,在之后计算过程中需要重复使用。就可以调用data1.cache接口将数据缓存在Storage Memory区域。在之后计算过程中框架就可以直接从Storage Memory区域读取数据。具体来说,调用cache接口并不会立刻将数据写入缓存之中,而只是设置了STORAGE_LEVEL这个标记为MEMORY_ONLY。后续在框架执行的过程中,当data1被计算得到时,框架会检查STORAGE_LEVEL这个标记,发现为MEMORY_LEVEL后会通过memory maneger将其存入Storage Memory之中。这个过程并不会造成内存拷贝复制。memory manager只是将指向数据的引用从Execution Memory移到Storage Memory之中,并更新两个内存区域内存总量。
    \item Execution Memory 用于框架计算过程。在计算过程中比如对于Shuffle操作,需要在内存中缓存一部分数据,再一次性写入文件系统之中,从而避免频繁磁盘IO导致的延迟。在计算的过程中数据也都存放在Executor Memory之中。比如上面所说的data1,在计算过程中一直使用Execution Memory。在计算结束之后根据STORAGE_LEVEL,如果为MEMORY_ONLY就会移到Storage Memory之中。
    \item User Memory 为应用程序使用的内存空间。应用程序创建的各种和Spark框架无关的对象都会存在User Memory之中。
    \item Reserved Memory 的目的是避免发生OOM(Out of Memory)错误。OOM问题的根本原因是
\end{enumerate}

\subsection{分布式弹性数据集RDD}
\subsection{作业执行调度模块}
\subsection{缓存数据管理模块}
\section{缓存管理模块分析}
\subsection{Shuffle缓存}
\subsection{Storage缓存}
\subsection{Unroll缓存}
\section{本章总结}




