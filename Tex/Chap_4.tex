\chapter{一种基于DAG作业运行时信息及拓扑结构的缓存替换策略的设计与实现}\label{chap:guide}
\section{Spark系统缓存替换策略分析}

在第二章已经详细分析过Spark默认的LRU缓存替换策略。是通过BlockManager管理RDD缓存数据、Shuffle数据、BroadCast缓存数据。本章重点关注RDD缓存数据。BlockManager内部有MemoryStore和DiskStore。分别用来管理内存缓存数据以及磁盘缓存数据。MemoryStore内部是通过LinkedHashMap存储block数据,LinkedHashMap是具有LRU特性的。

Spark框架使用简单的LRU缓存替换策略是有潜在的问题的。因为LRU策略仅仅考虑了数据访问的时间特性,在缓存空间不足的时候优先替换最长时间为访问的数据。这种方法它的优点是实现方法简单,时间复杂度低,在CPU Cache缓存场景是很合适的,因为Cache中数据大小一样,内存访问请求对与CPU来说也具有相同的重要性,因为从CPU的视角来看,不同的物理地址并没有什么不同。另一个方面,LRU缓存策略比较简单,在CPU物理器件内部比较容易实现。所以在CPU缓存场景下,LRU这种缓存替换策略是很适用的。但是对于Spark框架来说,LRU策略就有一些不足。因为不同的RDD具有不同的重要性,数据大小也不同。有的RDD数据是通过复杂计算,耗费了很长时间计算得到的。并且对于整个DAG计算图来说,不同的RDD的重要性也各不相同。如果对DAG中的关键路径的RDD节点,如果发生了数据丢失就会影响到整个DAG计算图作业的端到端执行时间。所以对LRU缓存替换策略进行进一步的改进,对于提升任务执行效率,减少任务端到端执行时间具有重要的意义。

\section{基于DAG作业运行时信息及拓扑结构的缓存替换策略的设计}

本文提出了一种基于DAG作业运行时信息以及DAG计算图拓扑结构的缓存替换策略。缓存管理模块会实时采集作业执行中的信息,比如作业执行时间,数据大小。并且根据DAG计算图的拓扑结构动态调整缓存优先级,提高缓存数据是使用效率并减少作业端到端执行时间。详细的说,本文提出了一种缓存数据优先级计算算法。该算法综合考虑DAG图中前驱作业执行状况,DAG图拓扑关系动态计算各个缓存数据的优先级。通过将重要数据的优先级变高,从而保证发生缓存数据替换的时候能够将从整个DAG时间来看优先级较低的数据替换删除,这样就可以将重要数据一直保存在内存缓存之中,这样在之后的作业执行过程中缓存数据就有更高的使用效率,从而可以提高整体作业的执行效率。

\subsection{根据DAG拓扑结构计算RDD优先级}

Spark应用程序会组成一个DAG计算图。DAG图中的节点为RDD数据,DAG图中的边为RDD之间的转换关系。不同的点也就是RDD数据大小各不相同,不同的边也就是RDD之间的转换关系的计算代价也各不相同。所以Spark应用的执行过程可以看作是对DAG图的遍历过程,端到端的执行时间为输入节点到输出节点的最短路径中最短的一条路径。这一条路径也就是DAG图的关键路径。所以缓存替换策略应该考虑整体DAG图的拓扑结构。对于关键路径中的RDD缓存数据应该尽量保存在缓存之中,应为如果将关键路径中的RDD数据替换出去,在之后的计算过程中如果访问到替换出去的数据,发现数据丢失,框架就就根据容错原理,根据拓扑结构重复计算丢失的数据,这样就会极大地增加端到端的执行时间。

DAG图的信息

\subsection{RDD计算代价}

\subsection{RDD访问次数}

\subsection{RDD数据大小}

\section{基于DAG作业运行时信息及拓扑结构的缓存替换策略的实现}
\section{基于DAG作业运行时信息及拓扑结构的缓存替换策略测试与性能分析}
\section{本章总结}